

% COLORS
\DefineNamedColor{named}{eclipsered}{rgb}{0.686,0.066,0.082}
\definecolor{tablecolor}{named}{eclipsered}


% CUSTOM COMMANDS AND ABBREVIATIONS
\newcommand{\todo}[1]{{\color{red} TODO: #1} \typeout{TODO TODO TODO #1}}
\newcommand\nobrkhyph{\mbox{-}}

\newcommand{\dice}[1]{{#1}}
\newcommand{\skill}[2][]{{\ifthenelse{\equal{#1}{}}{#2}{#2\,(\num[retain-explicit-plus]{#1})}}}
\newcommand{\modifier}[1]{{\num[retain-explicit-plus]{#1}}}
\newcommand{\dv}[1]{{DV\,#1}}
\newcommand{\sv}[1]{{SV\,#1}}

\def\eclipsephase{Eclipse Phase\xspace}
\def\ep{EP\xspace}


% Fancy red box for item lists
\newcommand*\itembox{\item[\color{eclipsered}\ding{110}]}



% For abbreviations such as i.e., e.g., …
% Also, omit final dot from each def.
\ExplSyntaxOn
\newcommand\latinabbrev[1]{
  \peek_meaning:NTF . {% Same as \@ifnextchar
    #1\@}%
  { \peek_catcode:NTF a {% Check whether next char has same catcode as \'a, i.e., is a letter
      #1.\@ }%
    {#1.\@}}}
\ExplSyntaxOff

\def\eg{\latinabbrev{e.g}}
\def\etal{\latinabbrev{et al}}
\def\etc{\latinabbrev{etc}}
\def\ie{\latinabbrev{i.e}}

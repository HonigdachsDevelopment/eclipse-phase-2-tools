
\section*{Psi}
\DefineNamedColor{named}{eclipsered}{rgb}{0.586,0.466,0.682}
\definecolor{tablecolor}{named}{eclipsered}



\begin{eptable}{ l | X }
   \rndheader{2}{Sleights}
   Psi-Chi & Always on.\\
   Psi-Gamma & Require action to activate and successful Psi-Test.\\
\end{eptable}

\bigskip


\begin{eptable}{ l | X }
   \rndheader{2}{Opposed Psi Test}
   Defender Wins Critical & Async locked out.\\
   Async Wins Critical & Double effect or damage, or half Psi armor.\\
   Async Fails Critical & \dv{1d6 + strain} \textbf{physical} damage.\\
\end{eptable}

\begin{itemize}
    \itembox Target opposes with \skill{WIL}. May go full psychic defense as complex action to gain \modifier{+30}.
    \itembox Target aware \textit{something} is happening every time it wins its Psi test.
    However, very unlikely it understands that targeted by Psi unless Async itself.
\end{itemize}


\bigskip

\begin{eptable}{ l | X }
   \rndheader{2}{Targeting}
   Multiple Targets & Single roll vs. multiple opposed. IR \modifier{+5} per target more.\\
   Animals & Partially sapient \modifier{-20}, non-sapient animals \modifier{-30}.\\
   Aliens & At least \modifier{-20}, might not work at all.\\
\end{eptable}



\bigskip

\begin{eptable}{ l | X }
   \rndheader{2}{Psi Range}
   Self & Only the Async.\\
   Touch & Physical contact. For touch attack \modifier{+20}, do dmg., same action.\\
   Close & Beyond \SI{10}{m} gives \modifier{-10} per meter.\\
   Psi vs Psi & Touch becomes Close, Close \SI{20}{m}.\\

\end{eptable}


\bigskip

\begin{eptable}{ l | X }
   \rndheader{2}{Duration}
   Constant & Always on.\\
   Instant & Immediate and permanent.\\
   Temporary & Lasts \skill{WIL} + 5 time units, as defined in sleight.\\
   Sustained & Requires concentration, \modifier{-10} to Async for duration.\\
\end{eptable}

\begin{itemize}
    \itembox Multiple sleights might be sustained at same time, each incurring additional \modifier{-10}.
    \itembox Async must also stay within range.
\end{itemize}

\bigskip

\begin{eptable}{ l | X }
   \rndheader{2}{Pushing Sleights}
    Increase Range & Touch becomes Close, Close \SI{20}{m}, Close (Async) \SI{30}{m}.\\
    Increase Effect & Any modifiers are doubled.\\
    Increase Power & Resisted by \skill{WIL}\,/\,2 instead of \skill{WIL}.\\
    Increase Penetration & Psi Shield reduced by half.\\
    Increase Duration & Double duration (temporary sleights only).\\
\end{eptable}

\begin{itemize}
    \itembox Pushing increases Infection Rating by additional \modifier{+5}. User automatically suffers \textbf{physical} strain of \dv{1d6\,+\,SM}.
    \itembox In addition to Pushing effects a regular Infection Test is made, which might inflict more physical strain, or cause other strain effects.
    \itembox Psi-Chi can be boosted as well. Boosting them lasts for \skill{WIL}\,+\,\SI{5}{min}.
\end{itemize}


\bigskip


\begin{eptable}{ X }
   \rndheader{1}{Infection Tests}
   Test made every time Psi-Gamma sleight activates (or pushing Chi).\\
   Target number equals current infection rating.\\
   Success is bad: produces \num{1} strain, each superior +\num{1} strain.\\
   Critical success even worse: \textit{Checkout time} or \textit{Interference}.\\
   In any case infection rating increases after test made.\\
\end{eptable}

\begin{itemize}
    \itembox \textbf{Checkout Time}: next long rest or unconscious situation infection may take over and acts with character without character knowing.
    \itembox \textbf{Interference}: When in future making some test, do an opposed \skill{WIL} vs. \skill[+30]{Infection} before that. If infection wins, player suffers critical failure. Preferably during dramatic situations.
    \itembox Each short rest reduces Infection Rating by \modifier{-10} (down to base). Each long rest resets Infection Rating to base.
\end{itemize}

\bigskip

\begin{eptable}{ l | X }
   \rndheader{2}{General Strain Effects}
   Breaking Point & Triggers existing disorder that lasts \dice{1d6} hours.\\
   Physical Damage & Suffer \dv{1d6\,+\,SM (Strain Modifier)} \textbf{physical}.\\
   Impulse Effect & Lasts \dice{1d6} min, resisting \skill{WIL} and \sv{1d6\,+\,SM}. \\
   Compulsion Effect & Lasts \dice{1d6} hours. If not acted upon by end \sv{1d6\,+\,SM}.
\end{eptable}

\begin{itemize}
    \itembox For actual Strain Effects roll on Async's Sub-Strain effects table. Sub-Strains might have additional effects not listed above.
\end{itemize}

\bigskip

\begin{eptable}{ l | X }
   \rndheader{2}{Architect Sub-Strain}
   1 & Breaking Point.\\
   2 & Physical Damage.\\
   3 & Compulsion (Creation of something weird).\\
   4 & Compulsion (Fascination about ALL aspects of thing or object, \textellipsis).\\
   5 & Compulsion (Hoard weird things, organs, \textellipsis).\\
   6 & Compulsion (Nest creation, hidden lair with traps, \textellipsis).\\
\end{eptable}

\bigskip

\begin{eptable}{ l | X }
   \rndheader{2}{Haunter Sub-Strain}
   1 & Breaking Point.\\
   2 & Physical Damage.\\
   3 - 4 & Hallucinations that seem totally real to you.\\
   5 & Haphephobia (Phobia of touch or being touched).\\
   6 & Impulse (Avoidance against type of sensory input).\\
\end{eptable}

\bigskip

\begin{eptable}{ l | X }
   \rndheader{2}{Predator Sub-Strain}
   1 & Breaking Point.\\
   2 & Physical Damage.\\
   3 & Impulse (Frenzy into melee combat).\\
   4 & Impulse (Respond to and accept any challenge or threat).\\
   5 & Impulse (Stand your ground, refuse to flee or surrender).\\
   6 & Compulsion (Feast on anything that recently died. Raw.).\\
\end{eptable}

\bigskip

\begin{eptable}{ l | X }
   \rndheader{2}{Stranger Sub-Strain}
   1 & Breaking Point.\\
   2 & Physical Damage.\\
   3 & Interference (like critical success on Infection Test).\\
   4 & Compulsion (Deceptive behavior, thrill of manipulating, lying).\\
   5 & Compulsion (Destructive behavior against self).\\
   6 & Compulsion (Test limits how far you can go in conflict).\\
\end{eptable}

\bigskip

\begin{eptable}{ l | X }
   \rndheader{2}{Xenomorph Sub-Strain}
   1 & Breaking Point.\\
   2 & Physical Damage.\\
   3 & Atavism, regress to animalistic behaviors.\\
   4 & Compulsion (Alien home environment for must be sought out).\\
   5 & Compulsion (Imprint on nearest person or object as mother or child).\\
   6 & Compulsion (Hunger, sniff, taste everything around).\\
\end{eptable}

\begin{itemize}
    \itembox Sub-strain effect list is gross oversimplification. Check rule book for detailed advice how strain acts and how to role play.
\end{itemize}
